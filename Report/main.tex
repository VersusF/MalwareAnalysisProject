\documentclass[a4paper]{IEEEtran}
\usepackage[utf8]{inputenc}
\usepackage{color}
\usepackage{graphicx}
\usepackage{mwe}
\usepackage{amsmath}
\usepackage{booktabs}
\usepackage{multicol}

\title{Malware Analysis Report\\\Large{``Sample2.exe''}}
\author{
    \begin{tabular}{lr}
        Contro Filippo &VR437055\\
        Martini Michele &VR437056
    \end{tabular}
}

\begin{document}
	\maketitle
	\section{Executive summary}
	The malware ``Sample2.exe'' shows itself as a calculator app, with the appropriate icon and the expected behavior. Once it is opened by the user there is a simple GUI to perform simple arithmetic operations. Under the hood, however, the malware deactivates the Windows security center, makes many POST request to various internet sites and infect many other files in different locations, from the local folder to the system application's executable.
	
	We analyzed the executable exploiting given virtual machine, which runs Windows 7 as operating system. %TODO Not sure if write this or whatsoever
	
	\section{Static analysis}
	We started file's static analysis using \emph{PEStudio}, already installed in the virtual machine, which allow us to verify many indicators of executable malevolous behavior. 
	First of all, we noticed an \emph{entropy} value higher than 7, therefore it's likely that the file has been packed or some of its sections are obfuscated.
	Software is also lacking of certification and compilation data is missing.\\
	
	\noindent The malware's code is made of the following sections:
	\begin{itemize}
		\item \textbf{.text}: the code in clear;
		\item \textbf{.data}: contains global variables;
		\item \textbf{.rsrc}: contains various resources, like images or icons;
		\item \textbf{.vmp0}: main section of the code, obfuscated to make reverse-engineering more difficult.
	\end{itemize}
	The last section is the larger with a file-ratio of 81.70\%, it has entropy higher than 7  and PEStudio marks it as blacklisted. Finally, above all, it's writable and executable.\\
	The presence of these sections allows us to suppose the code has not been packed and its high entropy is due to .vmp0 section's obfuscation. In order to validate this theory, we used \emph{PEID}, a software whose goal is to detect any possible packing.
	This tool provides three types of analysis: %TODO
	\begin{enumerate}
		\item Something
	\end{enumerate}
	
	%TODO Indicators and import analysis
	
	\vspace{1em}
	Regarding strings we notice that those in .text, .data and .rsrc sections are in clear; furthermore many of them are function calls or values used to modify registers.
	Then, there are thousands strings obfuscated and seemingly meaningless which probably belong to .vpm0 section. % thousands -> countless ?	
	
	
	\section{Dynamic analysis}
	\section{Reverse Engineering}
\end{document}