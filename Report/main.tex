\documentclass[a4paper]{IEEEtran}
\usepackage[utf8]{inputenc}
\usepackage{color}
\usepackage{graphicx}
\usepackage{mwe}
\usepackage{amsmath}
\usepackage{booktabs}
\usepackage{multicol}

\title{Malware Analysis Report\\\Large{``Sample2.exe''}}
\author{
    \begin{tabular}{lr}
        Contro Filippo &VR437055\\
        Martini Michele &VR437056
    \end{tabular}
}

\begin{document}
	\maketitle
	\section{Executive summary}
	The malware ``Sample2.exe'' shows itself as a calculator app, with the appropriate icon and the expected behavior. Once it is opened by the user there is a simple GUI to perform simple arithmetic operations. Under the hood, however, the malware deactivates the Windows security center, makes many POST request to various internet sites and infects many other files in different locations, from the local folder to the system application's executable.
	
	The whole analysis process has been done on the provided virtual machine, which runs Windows 7 as operating system and contains a bunch of tools to make both static and dynamic analysis. The Reverse engineering instead has been done on local machines.
	
	\section{Static analysis}
	We began the static analysis using \emph{PEStudio}\footnote{https://www.winitor.com/}, a tool which performs malware initial assessment, studying the headers and various information on the executable, which is never executed.
	This first analysis gave us many indicators of probable malicious behavior.
	First of all, we noticed an \emph{entropy} value greater than 7, therefore it's likely that either the file has been packed or some sections has been obfuscated.
	The executable is lacks also of certification and any information about the compilation is missing.
	
	\noindent The malware's code is made up of four sections:
	\begin{itemize}
		\item \textbf{.text}: the code in clear;
		\item \textbf{.data}: contains global variables;
		\item \textbf{.rsrc}: contains various resources, like images or icons;
		\item \textbf{.vmp0}: main section of the code, obfuscated to make reverse-engineering more difficult.
	\end{itemize}
	The last section is the larger with a file-ratio of 81.70\%, it has an entropy greater than 7  and PEStudio marks it as blacklisted. Moreover it is both writable and executable, which is often a malicious evidence, as it indicates self modifying code.
	
	The presence of these sections allows us to suppose the code has not been packed and its high entropy is due to .vmp0 section's obfuscation. In order to validate this theory, we used \emph{PEID}, a software whose goal is to detect any possible packing.
	This tool provides three types of analysis: %TODO
	\begin{enumerate}
		\item Something
	\end{enumerate}
	
	%TODO Indicators and import analysis
	
	\vspace{1em}
	Regarding strings we notice that those in .text, .data and .rsrc sections are in clear; furthermore many of them are function calls or values used to modify registers.
	Then, there are thousands strings obfuscated and seemingly meaningless which probably belong to .vpm0 section. % thousands -> countless ?	
	
	
	\section{Dynamic analysis}
	\section{Reverse Engineering}
\end{document}